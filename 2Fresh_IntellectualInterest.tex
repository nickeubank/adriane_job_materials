\documentclass[11pt]{article}


\usepackage{soul}
\usepackage[T1]{fontenc}
\usepackage{amsfonts, amsmath, amssymb}
\usepackage{multirow}
\usepackage{epsfig}
\usepackage{subfigure}
\usepackage{subfloat}
\usepackage{graphicx}
\usepackage{booktabs}
\usepackage{longtable}
\usepackage{verbatim, rotating, paralist}
\usepackage{enumerate}


\usepackage{natbib}
\usepackage{pdfsync}
\usepackage{latexsym}
\usepackage{amsthm}
\usepackage{mathabx}


\usepackage{stmaryrd}
\usepackage{mathrsfs}
\usepackage{dsfont}
\usepackage{parskip}
\usepackage{anysize, indentfirst, setspace}
\usepackage[right=1.7cm, left=1.7cm, top=2.5cm, bottom=2.5cm]{geometry}
\usepackage{epigraph}
\usepackage{appendix}
\usepackage{fancyhdr}


\renewcommand{\topfraction}{.85}
\renewcommand{\bottomfraction}{.7}
\renewcommand{\textfraction}{.15}
\renewcommand{\floatpagefraction}{.66}
\renewcommand{\dbltopfraction}{.66}
\renewcommand{\dblfloatpagefraction}{.66}


\pagestyle{myheadings}
\markright{ }


% header
\pagestyle{fancyplain}
\rhead{\hfill \small \emph{Adriane Fresh --- Intellectual Interest Synopsis --- Page \thepage}}
\cfoot{}
\renewcommand{\headrulewidth}{0pt}










%%  BEGIN DOCUMENT %%
%%--------------------------------%%

\begin{document}

\singlespacing
\setlength{\parindent}{0in}
\setlength{\parskip}{.15in}


% \thispagestyle{empty}



%%  HEADER  %%
%%----------------%%


\noindent \large{\textbf{\so{Intellectual Interest Synopsis}}} \\ \normalsize
\noindent Adriane Fresh \\
\noindent Re-appointment Review, AY2022-2023  \\


I am a scholar of political institutions and political economy.  I study how political elites respond to dramatic economic and institutional changes.  The goal of my research is to understand the nature of events and processes that can disrupt elites from power, and the strategies that elites employ to preserve their power.  This work is deeply motivated by an interest in processes of long-run historical persistence and the way in which elites and institutions interact in that process.  Below I provide an overview of my two research projects within this broad agenda.

\vspace*{.02in}
\textbf{My First Project (Dissertation)} \hspace*{.12in} A principle goal of my research is to understand how political elites respond to substantial economic change. On the one hand, we tend to think of economic change as potentially disruptive to a political equilibrium.  And yet, there is also substantial evidence to suggest that existing elites are well-positioned to capture the benefits of economic change and to persist in power.

To understand these potentially contradictory logics, I argue that studies of elite persistence in the context of economic change fail to account for both the economic interests of the elite, and the connections those elites have to the former elite via incumbency, dynasty, and social relationships.  Absent such a conceptualization and attendant measurement, it's not possible to understand the processes by which economic and political power reinforce one another, nor when they fail to do so.  With that conceptualization and measurement in mind, my research then argues that the industrial organization of economic change deeply shape how different groups in society are able to benefit from change.  This in turn affects the extent to which existing elites are able to persist in power.  The nature of change itself further interacts with the structure of political institutions and the broader international and informational environment to potentially privilege existing political elites in capturing change, or alternatively facilitate the rise of a new, challenger elite.

Empirically, I study two key moments of economic change in the British Isles---first, the early modern period of expanding overseas trade and growing market-oriented agriculture; and second, the Industrial Revolution.  In each case, I pair systematic measures of elite persistence and turnover with measures capturing various aspects of a transforming economy.  To do this, the work has relied heavily on systematic secondary source data collection paired with original archival materials, allowing me to trace processes at the individual level over the centuries-long temporal scale at which they unfold.  

The project makes a number of important contributions.  Existing work has increasingly viewed history as akin to a series of punctuated equilibria---disruptive events occur, and their effects largely persist.  In doing so, the work speaks not only by questions of how different sources of power---economic, social---convert into political power, but also how the composition of political power allows institutions to persist and the conditions under which we should instead expect them to change.

% But where the expanding economy privileged new social groups with new skills complimentary to the new type of trade---as it did in the Americas trades---elite turnover did take place. My findings indicate that these revolutions increased electoral competition and facilitated the entry of new economic interests into political power---but that these interests were represented by fundamentally new individuals and families.  Institutional changes like democratization hastened these transformations, but were not strictly necessary for them to occur. When we observe a change in the economic interests of political elites, it does not necessarily imply the rise of a new socio-economic group to political power.  my work makes a simple but critical observation about how we conceptualize elite persistence relative to turnover.  For example, the extent economic change's asset complementarity to the existing elite, and the ease with which change can be taxed, appropriated, or ``blocked'' entirely.

This project involves a core of (1) two working papers that are undergoing the review process, and (2) a book manuscript in progress; as well as (3) a set of working papers and in-progress articles that extend various theoretical and empirical aspects of the project.

\vspace*{.02in}
\textbf{My Second Project} \hspace*{.12in}   My second research project takes a comparative political economy perspective on questions about historical enfranchisement and contemporary strategies of disenfranchisement in the U.S., with a particular emphasis on racial conflict.

In this project, I'm particularly interested in how white elites in the U.S. South have contended with the exogenously imposed institutional change of the enfranchisement of Black people as a consequence of the 1965 Voting Rights Act.  As with my first project, I am concerned with the distributional consequences of change---however, in the context of enfranchisement, I'm interested both in the obvious shift in political power (including estimating that change with the best-available characterization of the counterfactual), but also how political change upset a social hierarchy built on, and reinforced by racial political exclusion.  I build on work that argues that enfranchisement was profoundly destabilizing for this hierarchy, and I evaluate the consequences in terms of social control---finding that enfranchisement increased the racially-differentiated use of the carceral state following enfranchisement.  The work firmly situates social control within the repertoire of elite tools for contending with disruptive change.

In addition, this work considers contemporary strategies of disenfranchisement---specifically, the choice by election administrators of where to locate polling places, and when, and under what conditions, to move them.  While journalistic accounts have emphasized that these choices are partisan and/or racially targeted, in turn depressing turnout, systematic investigations have been lacking.  My work finds that there is no evidence of systematic strategic manipulation, nor that the 2013 \emph{Shelby} Supreme Court decision abrogating Section 5 of the VRA increased race-based manipulation.  Moreover, polling place changes do not depress overall voter turnout.  Instead, voters substitute into other modes of voting, highlighting the value of early voting, another battleground in contemporary franchise wars.

% I've further investigated the extent to which elites used the institutions of the carceral state---police, the courts and the prison system---as a de facto ``New Jim Crow'' when the VRA removed their ability to use other strategies of disenfranchisement.  To evaluate these claims, we collected an original county-level dataset on mid-20th century incarceration-by-race for the former confederacy from archival corrections reports and inmate registers.  Using a series of difference-in-differences designs, we find that jurisdictions treated most intensely by Black enfranchisement were those that differentially increased race-based incarceration.

This project currently comprises (1) four published articles, (2) an additional working paper undergoing the review process, along with (3) a number of projects in-progress, including grant-funded work.



\end{document}
