\documentclass[11pt]{article}


\usepackage{soul}
\usepackage[T1]{fontenc}
\usepackage{amsfonts, amsmath, amssymb}
\usepackage{multirow}
\usepackage{epsfig}
\usepackage{subfigure}
\usepackage{subfloat}
\usepackage{graphicx}
\usepackage{booktabs}
\usepackage{longtable}
\usepackage{verbatim, rotating, paralist}
\usepackage{enumerate}


\usepackage{natbib}
\usepackage{pdfsync}
\usepackage{latexsym}
\usepackage{amsthm}
\usepackage{mathabx}


\usepackage{stmaryrd}
\usepackage{mathrsfs}
\usepackage{dsfont}
\usepackage{parskip}
\usepackage{anysize, indentfirst, setspace}
\usepackage[right=1.7cm, left=1.7cm, top=2.5cm, bottom=2.5cm]{geometry}
\usepackage{epigraph}
\usepackage{appendix}
\usepackage{fancyhdr}


\renewcommand{\topfraction}{.85}
\renewcommand{\bottomfraction}{.7}
\renewcommand{\textfraction}{.15}
\renewcommand{\floatpagefraction}{.66}
\renewcommand{\dbltopfraction}{.66}
\renewcommand{\dblfloatpagefraction}{.66}


\pagestyle{myheadings}
\markright{ }


% header
\pagestyle{fancyplain}
\rhead{\hfill \small \emph{Adriane Fresh --- Pandemic Impact Statement --- Page \thepage}}
\cfoot{}
\renewcommand{\headrulewidth}{0pt}










%%  BEGIN DOCUMENT %%
%%--------------------------------%%

\begin{document}

\singlespacing
\setlength{\parindent}{0in}
\setlength{\parskip}{.15in}


% \thispagestyle{empty}



%%  HEADER  %%
%%----------------%%


\noindent \large{\textbf{\so{Pandemic Impact Statement}}} \\ \normalsize
\noindent Adriane Fresh \\
\noindent Re-appointment Review, AY2022-2023  \\

It feels almost inappropriate to write a letter detailing the impact of the pandemic on my family given the millions of lives that have been lost over the two and a half years of the pandemic. Indeed, I am fully cognizant of how privileged I am to have come through this pandemic with my health and that of my family intact, to have kept my job, and to have been able to do that job without putting myself at substantial personal risk.  Nevertheless, I am submitting this statement because---in the context of professional review---I feel it is important to explain my own experience of the pandemic in order to provide important context for understanding my portfolio, even in the context of an additional year away from the tenure clock generously provided by Duke.

Like most two-career families with a young child---my partner is Asst. Research Professor in the Social Science Research Institute, also at Duke, and our daughter was a little over one-year-old when the pandemic began---my research productivity was severely disrupted by the pandemic primarily due to disruptions in childcare. Unlike the average two-career family with children, however, the impact on our family was particularly acute because we also had to contend with the fact that my partner has a significant chronic physical disability. The condition causes him constant pain, forces him to spend about half of his waking hours every day in bed resting---unable to read, work, or do anything else productive---and limits how long he can be out of bed to 3-4 hours at a stretch in most circumstances. As a result, he was already operating at his physical limit before the pandemic trying to balance his obligations as a parent, partner, and Professor, and so had no spare capacity to share the burden of the many disruptions to childcare that have occurred over the past several years.

Of course, we always knew that raising a family as two Duke professors would be challenging given his disability, but prior to the pandemic we had developed a series of strategies for coping with the challenges---we hired a nanny who agreed to care for our daughter Lyra whether she was healthy or sick on days we teach, we had two sets of grandparents who were eager (if distant---Seattle, Denver) to provide support when possible, and we had found a daycare in Durham with late-pickup (``late'' being 5pm).

But with the pandemic, nearly every one of these strategies fell through: our nanny was happy to help when Lyra had the sniffles in a pre-pandemic world, but when every cold had the potential to be Covid, it was out of the question; asking our relatively-older grandparents to fly to Durham from afar was a non-starter; and like most daycares, ours closed for a large portion of the pandemic in 2020, and even when ``open,'' it regularly but unpredictably shut down for extended periods every time a positive case was detected with a caregiver or child.   Moreover, because we had moved to Durham to start our positions in the late summer of 2019, we found ourselves dealing with this absent local family, friends, or established support network at the same time that we were navigating our first year as faculty, learning to teach remotely, and otherwise finding our way around a town we had only begun to figure out.

It's difficult to know the exact amount of time we went without childcare as we did not keep records at the time, but our full-time daycare closed in mid-March of 2020 and did not reopen until August. We were incredibly fortunate to find a nanny who was able to help us somewhat during that summer (at least when no one showed any signs of illness).  But, not only did we get fewer hours of coverage every week, but as a result we also ended up spending that summer with our nanny, her two-year-old son, our two-year-old daughter, and both my partner and I all trying to share a 1,500 sq. ft. house---including my partner and I both zooming and working from the same room---which was less than ideal from a productivity perspective.  As just one small example of the challenge, my partner uses voice dictation software to code and write as a function of his disability, while I prefer to quiet to work best.  Suffice it to say, noise-cancelling headphones only somewhat helped.

Even with a nanny, however, our access to childcare was far from regular. Throughout the pandemic, every time Lyra showed any signs of sickness---a \emph{very} common occurrence for a 2-year-old in the best of circumstances---we had to keep her home from daycare or ask our nanny to stay away until we could arrange for Lyra to get a PCR test and get back a negative result.   Moreover, even when our daycare reopened in August 2020, unpredictable disruptions remained a regular feature of life---in addition to having to keep Lyra home any time she showed any signs of illness, any time one of Lyra's classmates had a potential covid exposure, daycare would immediately close for a week so that everyone could quarantine and get tested. And when I tested positive for covid mid-pandemic, we were forced to not only isolate until she was no longer infectious, but then \emph{also} quarantine until my daughter and my partner were cleared.  Finally, even as adults, my partner and I were susceptible to regular illness (e.g., colds) that required scheduling PCR tests, keeping Lyra home from school, and waiting on results.  Again, we don't have records of exactly how frequently these disruptions it occurred, but looking at Lyra's medical records, by the summer of 2021 (when rapid tests first become readily available) our daughter had had eight PCR covid tests. And because of Lyra's age (she is now 4), even after my partner and I were vaccinated in April 2021, she remained unvaccinated until this summer, and we continued to experience daycare disruptions due to Lyra being sick and daycare closing down for exposures or positive tests, through January 2022.

In total, these disruptions amounted to months of scrambling to care for our daughter (without going to parks or organizing play dates where she might expose other children), arranging covid tests, and keeping up with our time-sensitive obligations. And because we were never able to fully keep up during those periods, we were also inevitably playing catch up in the days and weeks that followed each one of these childcare disruptions. As a result, over the past two and a half years, I constantly had to make difficult decisions about what to prioritize.  And I had to make those decisions within a household structure where I assume the majority of the basic childcare duties as a function purely of my physical capacity to do so.  And when forced to choose between my daughter, my partner, my students and my research, my research has almost always gotten short shrift despite my intense efforts to recognize my professional incentives and keep research from being a residual claimant to my time.

Of course, the impact of the pandemic on my research was not purely logistical; the pandemic also took a profound toll on our family's mental well-being. We have no family in Durham, and had only begun to make friends when pandemic lockdowns began in 2020, making the pandemic an especially isolating experience. In addition, my partner takes immune suppressants and his medical condition impacts his diaphragm, so our fear of Covid was especially acute, adding to our stress and precluding a lot of mild-risk social interactions that others chose to take.  Without oversharing, this impact was significant, resulting in mental health issues that resulted in a need for professional intervention.

I have also noticed various negative networks effects that have resulted from the pandemic.  The professional isolation has affected conference travel and presentations---those that did occur were virtual, with a decidedly different dynamic.  One consequence that I have felt is a lack of connection to potential collaborators.  In the past, collaborations have arisen in the context of informal interactions within networks, and those interactions have been few and far between over the past years. My partner is also one of my collaborators, and our correlated, nay common, household shocks have proved disruptive in a way that was difficult to anticipate pre-pandemic.

As noted at the top of this statement, I recognize nothing I have said here constitutes a significant burden within the context of the broader pandemic, nor do I presume that others in a similar situation have not faced similar, if not greater challenges.  I am writing this only to provide context for evaluating my research output.  Given my research productivity is being compared (at least implicitly) to the output of my peers, many of whom don't have children---and several of whom have said to me explicitly that they feel the pandemic has been ``one of the most productive periods of their careers''---I feel it is important for the review committee to understand some of the reasons the impact of the pandemic on our family likely differs from the impact it has had on others.

\end{document}
